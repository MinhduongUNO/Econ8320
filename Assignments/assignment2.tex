\documentclass[12pt, margin=.5in]{article}

\usepackage{fontspec}
\usepackage{hyperref}
\usepackage{natbib}

\defaultfontfeatures{Mapping=tex-text}
\setmainfont {Adobe Garamond Pro} % Main document font
\setsansfont {Gill Sans} % Used in the from address line above the to address


\pagestyle{empty}

\begin{document}
\vspace*{-6em}
\begin{center}
{\Large ECON 8320\   \ -- \ Tools for Data Analysis \\[.5em] Assignment 2 [25 points]
}
\end{center}

\setlength{\unitlength}{1in}

\hspace*{-4em}\begin{picture}(6,.1) 
\put(0,0) {\line(1,0){6.25}}         
\end{picture}
\hspace*{2em}
 
\begin{large}
In order to become more familiar with functions in Python, we will write some functions to solve various problems that will become important as we move further into data analytics. Write functions that can complete the following:

\begin{enumerate}
\item Find the largest (or smallest) number in a list of any size
\item Sort a list of numbers WITHOUT using the sort functions built into Python
\item Sort a list of words, ignoring upper and lower cases
\item Use the csv module (documentation at \url{https://docs.python.org/3/library/csv.html#module-csv}) to read a csv file, and store only the row with the largest number in the third column. Be sure to store the entire row, though!
\end{enumerate}

\end{large}


\end{document}